\section{Design considerations}
The development of our tool was guided by experts in urban sciences and
sociology whose main concern is to identify similar groups and how they evolved
over time, with as much detail as possible.


Practically, beyond the basis requirement that our method needs to be able to
work with non-geographically harmonized data, we divided the other requirements
into two different categories: \textit{Method}, the practical aspects of the
clustering method and classification, and \textit{Interface}, concerning the
visual representation of the information.

\begin{enumerate}
    \item[M1]{\textbf{Geographical information}: The clustering method needs to
    consider the available geographical information along with the data
    associated to the region.}

    \item[M2]{\textbf{Parameter configuration}: The configuration parameters for
    the clustering method should be configurable by the user.}

    \item[I1]{\textbf{Temporal evolution}: The evolution of each region over
    time can be inferred by its associated clusters. This evolution needs to be
    easily represented. }

    \item[I2]{\textbf{Cluster characteristics}:  The visual representation of
    each cluster should easily convey its relevant characteristics, including
    both geographic and content information.}

    \item[I3]{\textbf{Details of the changes}: Once a geographic region is
    selected, the interface should clearly convey how the associated information
    changed over time.}
\end{enumerate}
