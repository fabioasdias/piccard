\section{Discussion and limitations}
\label{sec:discussion}

Our objective was to leverage a network based data representation and
visualisation methods for the exploration of geographically inconsistent
region-based data. While we successfully replicated and corroborated results
from the literature, this method still has significant limitations. 

We removed the need for geographical harmonisation, but the method still
requires consistent variables across the years. Matching the variables can be
trivial for some aspects (Age), but challenging for others (Income). The
divulged income ranges vary over time and the actual values change due to
inflation. Since this is only a prototype, we matched few aspects, but a proper
demographic analysis would benefit from all available information. 

The limitation on the number of displayed clusters because of the limited number
of distinguishable colours was significant. While increasing the number of
clusters would further complicate an already complex analysis, it might be
warranted for some regions. Colour is a fundamental and intuitive tool for
information representation that can be coherently used across different plots,
so we opted to use it, even if in a limited way. With eight colours, there was
overlap between some clusters, the relevance gradient, and the colour
combination. 


The cognitive load on the user is significant, as we compromised simplicity for
accuracy. While other works labelled the clusters, as 'young urban',
'struggling', and so on~\citep{Delmelle2016,Delmelle2017}, we show the
statistical characteristics of the clusters, which are harder to interpret, as
the data may have subtle nuances that labels would otherwise hide. This also led
to a crowded interface, mitigated somewhat the use of pop-up panels and
collapsible sections. For some cities, especially if they are small and stable,
the panels can appear redundant, but each provide a different way to interact
with the information that can ease the exploration process for larger and
dynamic cities.