\section{Discussion and limitations}

Our objective was to leverage a graph based data representation and
visualization methods for the exploration of geographically inconsistent
region-based data. While we successfully replicated and corroborated results
from the literature, this method still has significant limitations. 


Removing the need for geographical harmonization greatly reduces the amount of
work necessary to explore demographic data, but the method still requires
consistent variables across the years. Matching the variables can be trivial for
some aspects (Age), but challenging for others (Income). The divulged income
ranges vary over time and the actual values change due to inflation. Some
variables were not considered in earlier censuses, such as Race in Canada, or
Hispanic population in the USA, hampering its use when they are available. Since
this is only a prototype, we matched few aspects, but a proper demographic
analysis would benefit from all available information.


While using one small map for each year leads to an easier visualization that
does not require interaction, it does not scale if more than five or six years
are considered. In this case, it might be interesting to replace the larger map
considering each year individually, along with a temporal control for
navigation. Indeed, including more years would likely lead  to a stronger
mixture of colors in the trajectory map, leading to a predominantly grey hue.


The limitation on the number of displayed clusters because of the limited number
of distinguishable colors was significant. While increasing the number of
clusters would further complicate an already complex analysis, it might be
warranted for some regions. Color is a fundamental and intuitive tool for
information representation that can be coherently used across different plots,
so we opted to use it, even if in a limited way. With eight colors, there was
overlap between some clusters, the relevance gradient, and the color
combination. 



Another limitation is the lack of control on how much the geographical
information will impact the clustering result. While the adopted method met our
needs for this work, a configurable control would add another dimension to the
exploration, allowing for more intra-cluster variance to obtain more 'compact'
clusters. We explored changing the number of content based augmented edges, but
this proved to be unreliable and hard to interpret. The~\emph{ClustGeo}
method~\cite{Chavent2017} can be a viable option for this, allowing a graph
based input and a hierarchical output, combined using a single mixing
parameter. Alternatively, one could cluster the changes~\cite{bian2018survey}
instead of the stable states.


There are also technological limitations, such as memory use on the
visualization client. To allow for changes on the CTs over the years, we use a
geographic file that contains all possible intersections, which can grow rather
large if the original city was expansive and contained several CTs, like NYC or
LA. However, the most significant technological limitation relates to parameters
that are not immediately interactive, such as the clustering configuration.
Since the clustering is computationally expensive and performed on the server,
which allows for cached results, some changes can take a few minutes to be
considered, removing any possibility of a continuous exploration. 


Indeed, the cognitive load on the user is already significant, as we compromised
simplicity for accuracy. While other works labelled the clusters, as 'young
urban', 'struggling', and so on~\cite{Delmelle2016,Delmelle2017}, we show the
statistical characteristics of the clusters, which are harder to interpret, as
the data may have subtle nuances that labels would otherwise hide. This also led
to a crowded interface, mitigated somewhat the use of pop-up panels and collapsible
sections. For some cities, especially if they are small and stable, the panels
can appear redundant, but each provide a different way to interact with the
information that can ease the exploration process for larger and dynamic cities.