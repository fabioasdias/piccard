\section{Discussion and limitations}

The objective of this work was to leverage a graph based data representation
with visual tools to allow for the exploration of geographically inconsistent
census data. While we successfully replicated and corroborated results from the
literature, this method still has significant limitations. 

Removing the geographical normalization/interpolation step greatly reduces the
amount of work necessary, but the method still requires consistent variables
across the years. Matching the fields of the public census can be trivial for
some aspects, like Age, but challenging for others, like Income. The divulged
income ranges vary over time, the actual value changes due to inflation and
other factors, and so on. Moreover, some fields were not considered in earlier
censuses, such as Race in Canada, or Hispanic population in the USA, hampering its
use when they are available. We matched some aspects, but a deeper demographic
analysis would greatly benefit from all available information.


Another limitation is the lack of control on how much the geographical
information will impact the clustering result. While the adopted method met our
needs for this work, a configurable control would add another dimension to the
exploration, allowing for more intra-cluster variance to obtain more 'compact'
clusters. We explored changing the number of content based augmented edges, but
this proved to be unreliable and hard to interpret. The~\emph{ClustGeo}
method~\cite{Chavent2017} can be a viable option for this, allowing a graph
based input and a hierarchical output, combined using a single mixing
parameter. Alternatively, one could cluster the changes~\cite{bian2018survey}
instead of the stable states.


There are also technological limitations, such as memory use on the
visualization client. To allow for changes on the CTs over the years, we use a
geographic file that contains all possible intersections, which can grow rather
large if the original city was expansive and contained several CTs, like NYC or
LA. However, the most significant technological limitation relates to parameters
that are not immediately interactive, such as the clustering configuration.
Since the clustering is computationally expensive and performed on the server,
which allows for cached results, some changes can take a few minutes to be
considered, removing any possibility of a continuous exploration. 


Indeed, the cognitive load on the user is already significant, as we compromised
simplicity for accuracy. While other works labelled the clusters, as 'young
urban', 'struggling', and so on~\cite{Delmelle2016,Delmelle2017}, we show the
statistical characteristics of the clusters, which are harder to interpret, as
the data may have subtle nuances that labels would otherwise hide. This also led
to a crowded interface, mitigated somewhat the use of pop-up panels and collapsible
sections. For some cities, especially if they are small and stable, the panels
can appear redundant, but each provide a different way to interact with the
information that can ease the exploration process for larger and dynamic cities.