\section{Related Work}
\label{sec:related}

Since our problem encompasses several fields, we divide this section into
specific sub problems: \emph{longitudinal demographic studies}, describing the
traditional tabular approach to longitudinal studies; \emph{data
representation}, elaborating how evolving geographic data can be represented for
processing; \emph{data clustering and regionalisation}, briefly reviewing
existing data clustering methods, geographic constraints and
regionalisation; and \emph{cluster characterisation}, articulating how clusters
can be visually summarised.

\subsection{Longitudinal demographic studies}
Census data is used not only to discover demographic
patterns~\citep{Firebaugh2016}, but to correlate demographic characteristics to
other measurements~\citep{diez1997neighborhood}. However, longitudinal studies
are rare, because they are difficult : \emph{"(...) One of the most challenging
and fascinating areas in spatial statistics is the synthesis of spatial data
collected at different spatial scales(...)"}~\citep{gotway2002combining}. While
census tract level data is readily available for the US since at least
1910~\citep{nhgis}, most studies consider the period between 1970 and 2010,
using pre-harmonised data from the Longitudinal Tract Data
Base~\citep{Logan2014} or the Neighborhood Change Database~\citep{ncdb}, despite
their inherent errors~\citep{Logan2016,Hallisey2017}. Similar harmonisation
efforts appear in other countries~\citep{Liu2015,Lee2015,Allen2018}. These
datasets have been highly significant for the field. Yet they also limit the
universe of data that can be used to study neighbourhood change, since any new
datasets would need to be similarly processed in order to be rendered compatible
with these sources. 

Another option considers the use of grid
data~\citep{Dmowska2017,Dmowska2018,stepinski2019imperfect}. Beyond the
potentially increased spatial precision, this approach does not require complex
harmonisation when new data is considered, if the grids are compatible. However,
demographic data is usually not available in this format, especially from older
sources. Additionally, the conversion from tabulation areas can introduce
significant errors.

Given these challenges, it is worth considering new alternatives. In this work,
we propose a novel methodology that entirely avoids the problems of geographical
harmonisation, considering each measurement using its actual geographic region.
It does not require regions to be consistent across time because they are
naturally represented as different entities. 

\subsection{Data representation}

Network based representation of geographic information is fairly well explored
in the literature, as a basis for topological methods for event
detection~\citep{Doraiswamy2014}, leveraging signal processing on
networks~\citep{shuman2013emerging,sandryhaila2013discrete} to find patterns and
outliers~\citep{Valdivia2015,Dias2015,Alce2018}. Networks are well suited to
represent trajectories as
well~\citep{VonLandesberger2016,Huang2016,chen2015survey}, allowing the use of
network visualisation methods~\citep{Vehlow2015,Beck2014}. Our proposed method
builds upon this literature. We leverage a network-based representation that
removes the rigidity in the measurement regions. Each region in time corresponds
to a different node. Instead of a collection of time-series, the data is
represented as a dynamic network. 

Networks have been used to represent census data for clustering purposes
~\citep{Dias2015,Setiadi2017}, but these works did not explore temporal
evolution, where they are particularly powerful. Networks allow a natural
representation of these inconsistent regions, with both spatial and temporal
connections. There are other possible representations that have similar
properties, but we adopted networks to allow the use of the vast existing
literature and methods.


\subsection{Data clustering and regionalisation}

Data clustering is one of the elementary processes for data analysis,
simplifying the data into a smaller number of homogeneous sets that can be
interpreted in the same way. There is no shortage of contributions for this
problem~\citep{Fahad2014}, since variations of it appear in almost all
scientific fields.


In geography, this problem is known as
\emph{regionalisation}~\citep{montello2003regions}, a rather old problem that
has been thoroughly explored, leveraging different mathematical tools, including
discrete topology~\citep{brantingham1978topological} and discrete
geometry~\citep{assunccao2006efficient}. Indeed, network-based methods are among
the current state-of-the-art~\citep{Guo2008Redcap,maxp}. However,
\emph{temporal} regionalisation is significantly less explored, especially in a
demographic context, arguably due to the difficulties in dealing with
unharmonised longitudinal data. Recent neighbourhood related applications rely
on k-means~\citep{jain2010data,Delmelle2016}, the Louvain method for community
detection~\citep{blondel2008Louvain,Thomas2012}, or, to a lesser extent, Self
Organising Maps~\citep{Delmelle2017,Ling2016,Dani2013_som_maxp}.


Since we adopted a network-based data representation and our objectives include
an interactive interface, we opted for an heuristic variation of the maximum
weighted matching algorithm called \emph{sorted maximal
matching}~\citep{markus2017}, because of its simplicity, customisability, and
fast computation times. This algorithm operates on weighted networks, where a
distance metric between the nodes is associated with the edges. We adopted a
distance based on the data associated with each node. The algorithm merges
clusters based on these distances, creating a hierarchy instead of a fixed
number of clusters. This hierarchical result is rendered by the interface,
allowing the user to change the number of clusters without reprocessing.
Changing the clustering algorithm would lead to different results, but any
hierarchical network clustering method, and distance metric, can be used in our
framework. 


\subsection{Cluster characterisation}
While visualisation has gained prominence as a crucial component of scientific
discovery, justification, and communication~\citep{tufte1998visual}, visually
representing evolving spatial data is a challenging old
problem~\citep{monmonier1990strategies,andrienko2003exploratory,ferreira2015visual,Zheng2016}.

Most geographic data is naturally bidimensional and maps work well in this
case~\citep{Zheng2016,ward2015interactive}, but the additional temporal
dimension cannot be so naturally represented. One straightforward option is to
leverage tridimensional plots~\citep{andrienko2014visualization,Tominski2012a},
but this can lead to visual obstructions or scaling problems unless a
tridimensional display device is used. A simpler, well adopted,
option is to display a map that corresponds to a subset of the temporal
information, allowing the user to change the time with an associated
control~\citep{Chen2017,Valdivia2015,Alce2018,Doraiswamy2014}. Small multiples
can be used~\citep{VonLandesberger2016}, but only when there are few temporal
snapshots. However, none of these options is suitable to represent many
variables at the same time.


Using data clustering, we can represent the region's cluster instead of all the
its variables~\citep{Alce2018,Valdivia2015,VonLandesberger2016}. While this
simplifies the geographic portion of the visualisation, it introduces the
problem of how to summarise the contents of each cluster. One traditional
approach is to use parallel coordinates plot~\citep{ferreira2015urbane}, but
they can get cluttered when representing similar clusters over several
variables. Further, for demographic applications, the clusters are usually
strongly characterised by a small subset of
values~\citep{Delmelle2016,Delmelle2017}. Therefore, in the proposed method, we
identify the variables that are most relevant to the characterisation of each
cluster. The distribution of values on that variable is then represented using a
boxplot, a well known statistical plot displaying basic properties of the
distributions.