\section{Related Work}

Since our problem encompasses several fields, we divided this section into
specific sub problems: \emph{longitudinal demographic studies}, describing the
traditional approach to perform longitudinal studies; \emph{data
representation}, exploring how evolving geographic data can be represented for
processing; \emph{Data clustering}, briefly reviewing existing clustering
methods; and \emph{cluster characterization}, exploring how the clusters can be
visually summarized.

\subsection{Longitudinal demographic studies}
Census data is used not only to discover demographic
patterns~\cite{Firebaugh2016}, but to correlate demographic characteristics to
other measurements~\cite{diez1997neighborhood}. However, longitudinal studies
are rare: \emph{"(...) One of the most challenging and fascinating areas in
spatial statistics is the synthesis of spatial data collected at different
spatial scales(...)"}~\cite{gotway2002combining}.

While CT level data is readily available for the US since 1910~\cite{nhgis},
most studies consider the period between 1970 and 2010, using pre-harmonized
data~\cite{Logan2014,nhgis}. Despite the inherent
errors~\cite{Logan2016,Hallisey2017}, this dataset became the standard source
for longitudinal demographic data, with similar efforts appearing in other
countries~\cite{Liu2015,Lee2015,Allen2018}. This result was significant for the
field, but it also restricts the usable data, since new datasets need to be
similarly processed.


Another option considers the use of grid data~\cite{Dmowska2017,Dmowska2018},
where small rectangular areas are used, in an approach similar to satellite
imagery. Beyond the increased spatial accuracy, this approach does not require
complex harmonization when new data is considered. However, demographic data is
usually not available in this format, especially from older sources, and the
conversion from tabulation areas can introduce significant errors.

In the proposed methodology, we avoid the harmonization by considering each
measurement using its actual geographic region. It does not require the regions
to be consistent across time because they are already represented as different
entities. 


\subsection{Data representation}
Most data is represented in tabular form, where the rows and columns have
coherent definitions. For example, consider a table with rain measurements over
time, with the rows representing different locations and the columns different
times. This representation can also be interpreted as a collection of
time-series, one for each location. Geographic data followed this format, only
including an additional field that describes the associated geographic area.
Following the example, the data would now represent the amount of rain for a
given region and time. As long each region remains the same, the data is
coherent and can be interpreted again as a collection of time-series.

In the proposed method, we remove the requirement for consistency in the
measurement regions by leveraging a graph-based representation, where each
region in time corresponds to a different node. Instead of a collection of
time-series, the data is represented as a dynamic graph. Graph based
representation of geographic information is fairly well explored in the
literature, as a basis for topological methods for event
detection~\cite{Doraiswamy2014}, leveraging signal processing on
graphs~\cite{shuman2013emerging,sandryhaila2013discrete} to find patterns and
outliers~\cite{Valdivia2015,Dias2015,Alce2018}. Graphs are well suited to
represent trajectories as
well~\cite{VonLandesberger2016,Huang2016,chen2015survey}, allowing the use of
graph visualization methods~\cite{Vehlow2015,Beck2014}. 

Graphs were used to represent census data for clustering purposes
before~\cite{Dias2015,Setiadi2017}, but these works did not explore temporal
evolution, where graphs are particular powerful as they allow a natural
representation of inconsistent regions, with both spatial and temporal
connections. Note that there are other possible representations that have
similar properties, but we adopted graphs to allow the use of the existing
literature and methods.


\subsection{Data clustering}
Data clustering is one of the elementary processes for data analysis,
simplifying the data into a smaller number of homogeneous sets that can be
interpreted in the same way. While there is no shortage of contributions for
this problem~\cite{Fahad2014}, most applications still rely on
k-means~\cite{jain2010data,Delmelle2016} and, to a lesser extent, Self
Organizing Maps~\cite{Delmelle2017,Ling2016}.

However, a method for geographic data analysis should not ignore the geographic
component of the data. One straightforward option, for agglomerative
methods~\cite{han2001spatial}, is to consider only nearby clusters for
merging~\cite{Chavent2017}, which can also be done for
k-means~\cite{soor2018extending}. Alternatively, the spatial distance could be
directly added to the inter-cluster metric~\cite{Chavent2017} via a mixing
parameter, which adds flexibility to the method, but introduces the problem of
finding the correct application-dependent values.

Indeed, one crucial step in most clustering algorithms is the definition of the
number of clusters. We sidestep this problem by considering hierarchical
methods~\cite{soille2012morphological}, where the result is not a partition of
the data, but a tree of partitions. This approach is interesting for interactive
methods, because it allows the user to change the number of displayed clusters
with minimal processing. Since our data is represented as a graph, one option
would be the watershed cuts algorithm~\cite{Cousty2009}, inspired by the well
known image processing segmentation and equally prone to over segmentation.
Considering that the processing time is also a relevant factor, we opted for an
heuristic variation of the maximum weighted matching algorithm called
\emph{sorted maximal matching}~\cite{markus2017}, which merges clusters based on
the weights of the edges between pairs of clusters.



\subsection{Cluster characterization}
Visually representing evolving spatial data is a challenging old
problem~\cite{monmonier1990strategies,andrienko2003exploratory,ferreira2015visual,Zheng2016}.
Most geographic data is naturally bi dimensional and maps work well in this
case~\cite{Zheng2016,ward2015interactive}, but the temporal dimension cannot be
so naturally represented. One straightforward option is to leverage
tridimensional plots~\cite{andrienko2014visualization,Tominski2012a}, but this
can lead to visual obstructions or scaling problems unless a tridimensional
display device is used. Animation can also be explored in some specific
cases~\cite{buschmann2014real}, but it is not a general approach. Glyphs can
also be used~\cite{seebacher2017visual,Andrienko2017}, but this may lead to
cluttering when many small regions are present. A simpler, well adopted, option
is to display a map that corresponds to a subset of the temporal information,
allowing the user to change the time with an associated
control~\cite{Chen2017,Valdivia2015,Alce2018,Doraiswamy2014}. Small multiples
can be used~\cite{VonLandesberger2016}, but only when there are few temporal
snapshots. However, none of these options is suitable to represent many
variables at the same time.


Using data clustering, we can represent the region's cluster instead of all the
its variables~\cite{Alce2018,Valdivia2015,VonLandesberger2016}. While this
simplifies the geographic portion of the visualization, it introduces the
problem of how to summarize the contents of each cluster. One traditional
approach is to use parallel coordinates
plot~\cite{ferreira2015urbane,LI2018,johansson2005revealing,guo2006visualization},
but these they can get cluttered representing similar clusters over several
variables. Further, for demographic applications, the clusters are usually
strongly characterized by a small subset of
values~\cite{Delmelle2016,Delmelle2017}. Therefore, in the proposed method, we
identify the variables that are most relevant to the characterization of each
cluster. The distribution of values on that variable is then represented using a
boxplot, a well known statistical plot displaying basic properties of the
distributions.