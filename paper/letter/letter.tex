\documentclass{article}
\usepackage[a4paper, total={6in, 8in}]{geometry}
\usepackage{tcolorbox}
\usepackage{amssymb}

\newcommand{\reply}[1]{\begin{tcolorbox}\noindent\textbf{Comment:}\\#1\hfill\end{tcolorbox}}
\newcommand{\didThis}[1]{\underline{#1}\marginpar{\checkmark}}

\title{Previous Reviews}

\begin{document}
\maketitle
A previous version of this paper was submitted to IEEE VAST 2018. While the full
text of the reviews is attached, we would like to overview the changes made,
leveraging the summary review:


\begin{itemize}
    \item{Evaluation: no evaluation of the proposed clustering methods (R3), the details
    of the domain expert interview are missing. (R1, R3)
    
    \reply{We did not introduce a new clustering method. The original paper
    contains such evaluations, we added that explicitly on the text. All the
    communication with the experts is included, verbatim, in the supplementary
    material, with the exception of Patrick Adler, which is a member of the
    Urban Genome Project, but did not collaborate directly in this work.}}

    \item{Design: the consideration and final choice of design are not clearly discussed.
    Specially, the color usage is confusing. (R2, R3, R4).
    
    }

    \item{Motivation: the authors fail to clearly demonstrate the motivation of their
    work, probably due to their writing style. (R2, R4)}

    \item{Presentation: no technical contribution clearly stated, writing style not
    qualified as a good research paper, paper organization can be improved (R2, R4)}
\end{itemize}

\section{Reviewer 1 - score 4/5}

\begin{itemize}
\item{\textbf{Paper type}\\Application}
\item{\textbf{Expertise}\\Expert}
\item{\textbf{Overall Rating}

    4 - Accept
    
    The paper should be accepted with some minor
    revisions.
    
    Once these have been completed it will meet the quality standard.}

\item{\textbf{Supplemental Materials}\\Acceptable}

\item{\textbf{Justification}

    the paper presents an interesting work on visualizing demographic evolution using
    geographically inconsistent census data. The strength of this paper is putting all
    the interesting pieces to present something an application that does show some
    useful insights into the data across time. The weakness of this paper is lack of
    anything new in particular and focused on a very specific type of demographic
    region (CT).
}
\item{\textbf{The Review}

    the paper presents an interesting work on visualizing demographic evolution using
    geographically inconsistent census data.

    Overall the paper reads well and it is easy to read and follow all the technical
    work. Most of the diagrams are useful. The paper tackles on a difficult problem of
    visualizing temporal data where each temporal piece may not be consistent with
    other pieces.

    The strength of this paper is putting all the interesting pieces to present
    something an application that does show some useful insights into the data across
    time. The weakness of this paper is lack of anything new in particular and focused
    on a very specific type of demographic region (CT). In summary, the work uses a
    spatio-temporal graph based approach and perform clustering based on the
    attributes of each node where each node represents a CT for a given year. The
    edges are added based on neighboring CT's or if they are same CT's but different
    years. This approach seems reasonable and has been adopted to model such problems
    in the past. The authors then perform clustering an introduce trajectories which
    all seems useful to the end-uses. Overall, the work seems useful to the community
    as it put together many ideas in the same application and it seems to work well
    with specific datasets and problem in hand.

    The weakness of the paper is very limited evaluation as mostly it was feedback
    from experts and the user-study lacked details. Also, most of the work is
    performed using CT as the region in census data, it would have been nice to see if
    this work is also applicable to other census regions as well.

    \reply{The specific region used for the analysis is irrelevant to the
    method, we could have used mixed regions for instance. We adopted CTs
    because it is the most finely grained data available including more than
    population count. Additionally, other census areas do not change as
    frequently as CTs. }

    \textbf{Revision Required:}
    Figure 2: while useful could use more detail and better explanation of the
    algorithm, it could do a better job in showing clustering and explaing the diagram
    since this is the figure that explains the entire paper.
}
\end{itemize}

\section{Reviewer 2 - score 2/5}
\begin{itemize}

\item{\textbf{Paper type}\\Application}

\item{\textbf{Expertise}\\Knowledgeable}

\item{\textbf{Overall Rating}

    2 - Reject
    
    The paper is not ready for publication in VAST / TVCG.
    
    The work may have some value but the paper requires major revisions or
    additional work that are beyond the scope of the conference review cycle to meet
    the quality standard. Without this I am not going to be able to return a score of
    '4 - Accept'.}

\item{\textbf{Supplemental Materials}\\Acceptable}

\item{\textbf{Justification}

    In this paper, the authors propose a method for longitudinal data analysis that
    avoids the geographical harmonization. The paper focuses on an interesting topic.
    The weakness of the paper is that the motivation and design requirements are not
    clearly demonstrated, and the design process is not justified.}

\item{\textbf{The Review}

    In this paper, the authors propose a method for longitudinal data analysis that
    avoids the geographical harmonization. Specifically, they adopt an improved
    clustering algorithm and a graph based visualization to help users identify
    clusters and the demographical details of each cluster. Though this paper is
    interesting to read, it still has several issues to be addressed. The authors fail
    to clearly demonstrate the motivation and the design requirements of the proposed
    work. A number of important details are also missing such as the discussion about
    design choices. In addition, the paper is not easy to follow in many sections such
    as the description of the teaser. I feel these issues can be hardly fully
    addressed in a minor revision and I suggest the authors to keep improving the
    work.

    My first concern is that the authors fail to clearly demonstrate the motivation of
    their work. The authors mentioned that longitudinal analysis of census data is
    missing and its first step is geographical harmonization that is difficult.
    Therefore, they propose a method that analyze geographical data without the need
    of geographical harmonization. However, they fail to demonstrate that why the
    longitudinal analysis of census data is needed and important. In addition, they
    also fail to illustrate why it is important and challenging for omitting the
    process of geographical harmonization in the data analysis. The authors need to
    explicitly describe the motivation of the proposed work.

    In addition, the authors are suggested to clearly discuss about the design
    considerations. For example, “Details of the changes” is not clear enough. Many
    questions can be asked: What kind of information the authors want to address for
    the changes? The geographic information or the content information or both? The
    authors need to clearly describe the design considerations.

    Moreover, the design choices are also not discussed. The authors need to discuss
    how they make the final design decisions and if there are any alternative design
    choices. For example, the authors use stacked bar charts to encode the information
    of each aspect, in which the rectangle width represents the percentage of the
    corresponding variable. Many questions can be asked: Why the author choose the
    stacked bar charts? Do they consider other design choices such as pie charts? Why
    they use the length of rectangle to represent percentage? Do they consider color
    saturation or hue? The authors need to clearly discuss the reasons they finalize
    the current design.

    The writings are hard to follow sometimes. For example, the description of the
    teaser is really difficult to understand. Many questions can be asked: What does
    each part represent in the figure? How to tell “While the whole city follows this
    trend”? What regions are “these regions”? What are the “relevant clusters” and why
    they are “visible”? The authors need to clearly explain the figures and improve
    the writing.

    In summary, the authors are suggested to revised the paper and improve their work
    to address the above issues.
}
\end{itemize}

\section{Reviewer 3 - score 3/5}
\begin{itemize}

\item{\textbf{Paper type}\\Application}

\item{\textbf{Expertise}\\Expert}

\item{\textbf{Overall Rating}

    3.5 - Between Possible Accept and Accept}

\item{\textbf{Supplemental Materials}\\Acceptable with minor revisions (specify
revisions in The Review section)}

\item{\textbf{Justification}

    This paper targets an interesting research question on analyzing the evolution of
    geographically inconsistent data. The major technical contributions of this paper
    are
    \begin{enumerate} 
        \item{A graph based spatio-temporal clustering method;}
        \item{an enhanced box plot visualization to better support comparison;}
        \item{the heatmap-like map view showing the spatial information of clusters.}
    \end{enumerate}
    \noindent Another big bonus is that the authors provided an interactive demo
    online. Still, there are several issues on 
    \begin{enumerate} 
        \item{the lack of evaluation of the proposed clustering method;} 
        \item{the design choice, mainly for the color usage;}
        \item{the lack of details of expert review process.}
    \end{enumerate}
    \noindent Detailed reviews can be found in the review section.}

\item{\textbf{The Review}

    In this paper, the authors proposed a visual analytics system to explore and
    analyze the evolution of geographically inconsistent data. Several visualization
    techniques, such as enhanced box plots, Sankey diagram, heatmap-like map
    visualization with small multiple settings, are used in the system. In general,
    the author tackles on an interesting and not easy research question, and the paper
    is well written and easy to follow. The case studies and expert interviews have
    clearly shown its usefulness and effectiveness.

    The major contributions have been mentioned above in justification section, here
    list the details of the major concerns.
    \begin{enumerate}

    \item{There is no proper evaluation of the proposed clustering methods. 

    In Section 4.2, the authors proposed a graph based geographic content clustering,
    which considers spatial, temporal, and similarity information among regions. Here,
    it will be worth to conducting comparisons with some traditional methods, for
    example, k-means purely based on histogram distances or weighted distance by
    considering spatio-temporal information (specific for the first one, since I think
    there is a high correlation of spatio-temporal information with the histogram
    distance).
    \reply{TODO}
    }

    \item{The color usage for the whole visual analytic system is sometimes confusing. 

    In the application, the authors use colors in different ways across
    visualizations. In the Sankey graph, the color is used to encode the categories.
    In the map view, additional opacity is used here (and blending is optionally used)
    together with the same color encoding as the one in the Sankey graph. From the
    right figure in Figure 6, it is clear that the color is not distinguishable (this
    is also related to the clustering itself since the three clusters are too small
    compared to the others). What makes it worse is the color usage in the box plot
    view to encode the relevance, which is very confusing even with the legend shown
    on top since there is some color overlap with the categorical encoding (e.g., the
    green and grape color).

    \reply{Colour assignment was indeed challenge for this work, mostly because
    it needs to be used, but there isn't a definitive way to actually do it.
    That was also the reason for artificially limiting the maxium number of
    clusters to eight. There is indeed overlap between the colours used for
    relevance and the clusters, mostly because the clusters used virtually all
    of the colour space. However, we believe that their different meanings are
    clear, since none of the experts, nor the other reviewers, mentioned any
    confusion in this specific aspect. It is not ideal, but we believe it is a
    acceptable approximation.}}

    \item{The details of the domain expert interview is missing in the paper.  
    I really appreciate that the authors attached the detailed expert review comments
    and the online demo is available. However, in the main paper, there is a very
    little description of how the expert interviews were conducted. Since there are
    still plenty of space, I would suggest the authors to add more details in Section
    6.}
    \end{enumerate}
    \textbf{Minor issue}:
    The demo video can be improved by providing a more comprehensive walkthrough of a
    case study instead of fragmented videos to explain the functionalities.
}
\end{itemize}

\section{Reviewer 4 - score 2/5}

\begin{itemize}

\item{\textbf{Paper type}\\Technique}

\item{\textbf{Expertise}\\Expert}

\item{\textbf{Overall Rating}

    2 - Reject
    
    The paper is not ready for publication in VAST / TVCG.
    
    The work may have some value but the paper requires major revisions or
    additional work that are beyond the scope of the conference review cycle to meet
    the quality standard. Without this I am not going to be able to return a score of
    '4 - Accept'.}

\item{\textbf{Supplemental Materials}\\Acceptable with minor revisions (specify revisions in The Review section)}

\item{\textbf{Justification}

    This paper introduces a combined graph clustering and geospatial visualization
    technique to illustrate the city census data that is potentially inconsistent over
    the space dimension. The technique is generally OK, though not entirely new, and
    can be said as appropriate for the problem attacked. The major strength is the
    detailed case studies on two north American cities and it is appreciated that the
    authors attach a full record of their expert study. On the other hand, the paper
    in its current shape suffers significantly from presentation issues, i.e., beyond
    minor grammar errors, most technical content would benefit from a thorough re-
    written. There is no explicit contribution statement. It seems the contribution on
    visualization is rather limited. The authors are also encouraged to re-submit to a
    GIS venue.

\item{\textbf{The Review}

    This paper proposed a framework to visualize the spatiotemporal population
    clusters detected from the census data. A graph representation is built where the
    nodes are fine-grained census tracts (CT) and edges are computed from the
    similarity of census attributes of related CTs. The graph is clustered by an
    existing watershed cut algorithm and then undergo a minor refinement. The derived
    spatiotemporal clusters are displayed on a geospatial map and their dynamics over
    time are visualized in a standard Sankey diagram. The work is well evaluated with
    census data in LA and Toronto. The feedbacks from five experts are also reported.
    While the technical side of this work does have some merit, there are at least two
    major flaws or technical incompleteness in the current presentation.

    First, I missed the technical contribution in this paper. In the first paragraph
    of Sec. 4, it is said ``beyond the main contributions'', what are these
    contributions? In fact, I see rather limited contribution in the visualization
    side. Besides the map design, the only ingredient of information visualization
    seems to be the Sankey diagram. I look at this diagram multiple times and could
    not sense what is different from a basic D3 example available on the website. The
    analytics part is similar, the SMM algorithm is not new, and the graph
    representation and clustering seems straightforward. There are little details
    discussed (in fact, should be some!). For example, why to connect the adjacent or
    overlapping CTs under different time affinities, how the clustering parameter is
    chosen? What are the nature of the resulting clusters? To this end, what is the
    specialized design for inconsistent census data?

    \reply{Following the guidelines, an application paper doesn't need to
    include new techniques. Indeed, the objective of this work was to leverage
    "commonplace" VIS methods to provide a more efficient solution to a common
    issue in another field. Indeed, we provide a viable alternative to time
    consuming and challenging problem that was believed to be unavoidable.}}


    Second, the writing style and paper organization is hardly a qualified
    research paper on visualization. Many key technical content are missing. The
    introduction and related work are fairly short, thus do not motivate this
    work well. For example, in the intro, I would like to see, what is the
    research problem, what are the challenges, and how you solve it with
    exquisite designs and/or algorithms. The authors only write one plain
    sentence there: ``We use a graph based spatio- temporal representation,
    combined with an improved clustering algorithm'', could be used for 100+
    papers on the similar topic. The related work cites an appropriate number of
    literature, but fails to discuss their pros and cons in order to position
    this work. The entire cluster of work by Andrienko et al. are missing while
    it is said: ``it is reasonable to argue that the availability and quality of
    the dataset provided by the Taxi and Limousine Commission of New York
    contributed to its centrality in this field''. The most related work on
    visualizing census data is also missing. In the visualization design part
    (Section 4), the authors spend several paragraphs on the coloring choice,
    which is good. Only one sentence is used to describe the Sankey diagram,
    which might be the only information visualization design applied in this
    work. Note that the graph representation has been the highlight of this
    work. No design rationale is discussed here. For example, what if I use a
    simple alternative of multiple/stacked line charts to represent the dynamics
    of spatiotemporal clusters? In the evaluation part, while I appreciate the
    detailed case study on LA and Toronto, the user feedback seems to be
    problematic given that the users are only given the video, not the
    interactive tool to try themselves! From the preview video, it is hard to
    imagine a real system exists for this work.

    \reply{Unfortunately the reviewer doesn't mention which is the "most related
    work". I did review some works from Andrienko's group, after all they are
    fairly known for this, but some were not included because they are not
    relevant to the current problem. TODO I included more text expliciting
    exactly what was accomplished in this work, which hopefully will make that
    distinction clearer. \\ As the others reviewers mentioned, the tool was
    available not only for the experts, but for the reviewers as well, on an
    anonymous site.}

    There are some other minor issues which the authors can address in any later
    versions.

    \begin{itemize}
        \item{The design consideration of M1 \& M2 are too general, could be
        said for most urban visual analytics methods}
        \item{Last paragraph of Section 4.2, why augment two edges, any quantitative
        evident/support from data?}
        \item{Figure 4 could be put at page 4 for better viewing experience}
        \item{Section 4.6, Sect. to Sec., on top left to on the top left}
        \item{It will be helpful to put a sample census data in the paper, e.g., a table}
        \item{The Sankey diagram in Figure 6 is cluttered, any solution?}
    \end{itemize}

    Overall, the authors have studied a valid problem with an appropriate
    solution and well-conducted case study, but the contribution on
    visualization is vague and many key technical content is missing. The
    writing quality and depth still have much room to improve. For these
    reasons, I return a reject score for now. A last point, if there is little
    new contribution on visualization, would this paper fit better to a GIS
    venue? 
    
    \reply{We considered a GIS venue for this work, but the technical part is
    fairly advanced to a general public, requiring significant tangents to
    explain all the involved concepts. Further publications on applied geography
    and demographics venues are planned in the scope of the Urban Genome
    Project, which will cite this publication for the technical details.}}

\end{itemize}

\subsection{The Summary Review}

\textbf{Summary Rating}

    Possible Accept
    
    The paper is not acceptable in its current state, but might be made
    acceptable with significant revisions within the conference review cycle.
    
    If the specified revisions are addressed fully and effectively I may be able
    to return a score of '4 - Accept'.

\section{The Summary Review}
    The reviewers have divided opinions on this paper.

    On the positive side, all reviewers recognize that the authors studied an
    interesting research problem and proposed appropriate/valid solutions to
    that problem (R1, R2, R3, R4). Some finds the proposed techniques to have
    enough contribution (R3), while some others comment these techniques as lack
    of novelty (R1, R4). Most reviewers are convinced on the usefulness of the
    approach as demonstrated in the application of the case studies (R1, R3,
    R4). R3 especially likes the released online demo which helps to grasp the
    main functionality of the technique.

    On the negative side, several concerns are raised and suggestions for
    revision are made, which the authors must address within the conference
    cycle if the paper is conditionally accepted.

    \begin{itemize}
    \item{Evaluation: no evaluation of the proposed clustering methods (R3), the details
    of the domain expert interview are missing. (R1, R3)}

    \item{Design: the consideration and final choice of design are not clearly discussed.
    Specially, the color usage is confusing. (R2, R3, R4)}

    \item{Motivation: the authors fail to clearly demonstrate the motivation of their
    work, probably due to their writing style. (R2, R4)}

    \item{Presentation: no technical contribution clearly stated, writing style not
    qualified as a good research paper, paper organization can be improved (R2, R4)}
    \end{itemize}

    More detailed comments can be found below.

    After discussion, we conclude that a borderline summary rating is appropriate. We
    do hope the authors can address all of our concerns in the conference cycle
    through significant efforts.

\end{document}