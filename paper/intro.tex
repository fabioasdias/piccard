% General context + motivation
\IEEEPARstart{U}{rban} sciences are blooming thanks to a renewed interest in
understanding and improving the urban environment. Visual analytics is following
this trend, fueled by new public datasets that encompass progressively more of
our daily lives~\cite{Chen2017}. There is no shortage of methods to explore
mobility patterns~\cite{VonLandesberger2016}, social media~\cite{Chen2017},
traffic~\cite{chen2015survey}, and so on, providing experts, planners, policy
makers, and the general population with deeper insights about their cities.


These new datasets usually contain GPS coordinates for the records, leading to
\emph{point-based} data. Combined with the corresponding timestamps, this data
is easily suitable for longitudinal analysis. But most demographic datasets are
\emph{region-based}, where the measurements are associated with pre-defined
regions, not only for an additional level of privacy protection, but because
some measurements only make sense over a defined area. Census data is a classic
example of this format, with datasets available from 1790 onwards for the
US~\cite{nhgis}. Despite this unmatched temporal availability, longitudinal
analyses of census data are often restricted in time, especially when smaller
tabulation areas are considered, such as census tracts (CT) or dissemination
areas, which evolve to reflect changes in population density, leading to
geographic inconsistencies across time, and the traditional time-series based
approach is no longer viable. 


However, these analyses are necessary to understand the urban environment.
Indeed, two different regions can have similar average income for a given year,
while one is experiencing a process of economic improvement and the other one
impoverishment. Quite obviously, a single snapshot cannot be used to identify
gentrification, migration, education changes, or any of the relevant processes
that happen over time.


To overcome these inconsistencies, the traditional approach is the
\emph{geographical harmonization} of the data, the interpolation of the
measurements into a common set of
regions\cite{Logan2014,Hallisey2017,Allen2018}, so that each variable can be
represented using time-series. This is laborious work that inevitably introduces
some amount of error~\cite{Logan2016}, even when additional data is
provided~\cite{eicher2001dasymetric}. Nevertheless, this step is considered
mandatory in the current literature: \emph{"(...) tract-by-tract comparison is
not possible unless data from 2000 is interpolated to 2010 boundaries
(...)"}~\cite{Dmowska2017}.



The main contribution of this application paper is an visualization-based
alternative to the geographical harmonization, a combination of established
graph based processing and information visualization techniques allowing
tract-by-tract comparison, the identification and visualization of patterns of
demographic evolution without geographic harmonization, effectively removing one
of the most challenging problems in longitudinal demographic analysis. We also
include illustrative scenarios and our prototype is available at
\url{http://uoft.me/piccard}, including more than fourty regions in the US and
Canada. The source code is publicly available at
\url{https://github.com/fabioasdias/piccard}.

